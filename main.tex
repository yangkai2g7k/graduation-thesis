\documentclass[twoside,numberorder]{csbachelor}
%==============================================================
%==============================================================

\usepackage{url}
\usepackage{subfigure}
\usepackage{pdfpages}
\usepackage[numbers]{natbib}
\citestyle{IEEEtran}


%一些全局工具的定义
\DeclareMathOperator*{\argmin}{arg\,min}
\DeclareMathOperator*{\argmax}{arg\,max}

%==============================================================
%==============================================================
\begin{document}
%==============================================================
%==============================================================

  %论文题目:{中文}{英文}
  \zjutitle{基于Dynamic Memory Network的问答系统的优化与应用}%
           {}
  %作者:{中文姓名}{英文}{学号}
  \zjuauthor{杨凯}{Yang Kai}{3130000495}
  %指导教师:{导师中文名}{导师英文名}
  \zjumentor{蔡亮}{Cai Liang}
  %个人信息:{年级}{专业名称}
  \zjuinfo{2013级}{软件工程}
  %学院信息:{学院中文}{学院英文}
  \zjucollege{计算机科学与技术}{College of Computer Science and Technology}
  %日期:{Submitted Date}
  %\zjudate{2017年3月20日}

%==============================================================

  %封面
  %%============================================================
%% 中文封面

\thispagestyle{empty}

\vspace{5mm}

\begin{center}
   \includegraphics[width=108mm]{images/zjdx}
\end{center}

\centerline{\heiti\erhao\textbf{本科生毕业论文}}
\centerline{\heiti\erhao\textbf{开题报告}}
\vspace{4mm}

\begin{center}
  \includegraphics[width=35mm]{images/standxb}
\end{center}

\vspace{25mm}

% {\hspace{16mm}\songti\sanhao\bfseries 题目:
%   \hspace{2mm} \begin{minipage}[t]{98mm}\linespread{1.1}\uline{\zjutitlec}\end{minipage}}

\vspace{7mm}

\begin{tabbing}
     \= \songti\sihao 题\hspace{10mm}目: \= \underline{\makebox[12cm]{\sihao\zjutitlec}} \\[2mm]
    \> \songti\sihao 姓\hspace{10mm}名: \= \underline{\makebox[12cm]{\sihao\zjuauthornamec}} \\[2mm]
    \> \songti\sihao 学\hspace{10mm}号: \> 
    \underline{\makebox[12cm]{\sihao\zjuauthorid}} \\[2mm]
    \> \songti\sihao 指导教师: \> \underline{\makebox[12cm]{\sihao\zjumentorc}} \\[2mm]
    \> \songti\sihao 专\hspace{10mm}业: \= \underline{\makebox[12cm]{\sihao\zjugrade\hspace{3mm}\zjumajor}} \\[2mm]
    \> \songti\sihao 学\hspace{10mm}院: \> \underline{\makebox[12cm]{\sihao\zjucollegec}}
\end{tabbing}


%%============================================================
% empty page for two-page print
\ifthenelse{\equal{\zjuside}{T}}{%
  \newpage\mbox{}%
  \thispagestyle{empty}}{}
  %诚信承诺书
  %% mentorassign

\newpage
\thispagestyle{empty}

\begin{tabbing}
\hspace{5mm}\songti\sihao 一、题目:\underline{\makebox[12cm]{基于Dynamic Memory Network的问答系统的设计与实现}}
\\ \\
\hspace{5mm}\songti\sihao 二、指导教师对开题报告、外文翻译和文献综述的具体要求:
\end{tabbing}
\begin{itemize}
\item 要求查阅相关的文献10篇以上(外文不少于5篇)。
\item 翻译的外文文献必须是研究性的论文,并且与论文主题直接相关.
\item 文献综述应包括国内外现状、研究方向、存在问题、参考依据等方面情况。
\item 译文和文献综述字数要求各3000字以上,开题报告字数为3500字以上。
\item 开题报告的内容应包括:
\begin{enumerate}
\item 主要研究内容、目的和意义;
\item 有关的国内外研究状况;
\item 课题难点和拟解决的关键问题;
\item 拟取的研究方法及其可行性、预期达到的目标;
\end{enumerate}
\end{itemize}
\vspace{60mm}

\begin{tabbing}
\hspace{80mm}\songti\xiaosi 指导教师(签名):
\\ \hspace{90mm} \songti\xiaosi 年 \hspace{5mm} \songti\xiaosi 月 \hspace{5mm} \songti\xiaosi 日
\end{tabbing}

\ifthenelse{\equal{\zjuside}{T}}{%
  \newpage\mbox{}%
  \thispagestyle{empty}}{}

  %考核
  %考核
\thispagestyle{empty}
{
\begin{center}
\stfangsong\sihao 毕业论文开题报告、外文翻译和文献综述考核
\end{center}
}
{\songti\sihao 导师对开题报告、外文翻译和文献综述评语及成绩评定:}
\\
\\  论文的文献综述部份对自然语言问答技术以及 Dynamic Memory Network的现状、研究方向等内容进行了分析,文献综述与论文的方向一致。开题报告对论文工作的主要内容、研究计划、预期进展等方面进行了相关介绍,外文翻译的术语翻译正确,符合要求,同意进入毕业设计的下一阶段。\\
\\
\\

{
\hspace{3cm} \songti\xiaosi
\begin{tabular}{|c|c|c|c|}
    \hline
    成绩比例 & \parbox[t]{4em}{开题报告\\[-1.5em]占(20\%)} &
               \parbox[t]{4em}{文献综述\\[-1.5em]占(10\%)} &
               \parbox[t]{4em}{外文翻译\\[-1.5em]占(10\%)} \\

    \hline
    分值   & & &  \\
    \hline
\end{tabular}
}
\begin{flushright}
    导师签字\;\underline{\hspace{4em}}\\
    年 \quad 月 \quad 日
\end{flushright}

{\songti\sihao 答辩小组对开题报告、外文翻译和文献综述评语及成绩评定:}
\vspace{4cm}

{
\hspace{3cm} \songti\xiaosi
\begin{tabular}{|c|c|c|c|}
    \hline
    成绩比例 & \parbox[t]{4em}{开题报告\\[-1.5em]占(20\%)} &
               \parbox[t]{4em}{文献综述\\[-1.5em]占(10\%)} &
               \parbox[t]{4em}{外文翻译\\[-1.5em]占(10\%)} \\

    \hline
    分值   & & &  \\
    \hline
\end{tabular}
}
\begin{flushright}
    答辩小组负责人(签名)\;\underline{\hspace{4em}}\\
    年 \quad 月 \quad 日
\end{flushright}


\setcounter{page}{1}
%==============================================================
%这部分不需要自己修改。

  %目次页
  \tableofcontents
  \thispagestyle{empty}
  %\chaptermark{目录}
  %\addcontentsline{toc}{chapter}{目录}

  %\mainmatter

%==============================================================

  %\chapter{本科毕业论文开题报告}
\setcounter{page}{1}
{\sanhao\heiti\filcenter \centerline{本科毕业论文开题报告}}
\section{课题背景}

人类语言具有特殊的结构,能够方便的传达意义,自然语言处理(NLP)就是研究人类语言的一个学科。而问答系统(Question Answering)的NLP问题之一。NLP中的很多问题都可以转换为QA问题(例如:机器翻译、情感分析、实体标记等)。随着智能语音助手(例如:苹果的Siri、谷歌的Google Assistant和微软的Cortana)的流行,
QA问题更是被广泛的予以关注。而QA问题本身是一项很复杂的NLP任务。想要对问题做出回答,需要具有分析一段文字,进行归纳总结的能力。\\
传统的解决QA的方法通常依赖于固定结构的知识数据库,而这些数据库通常是由特定领域的专家人工搭建的。当进行问题回答时,首先解析问题,然后将问题转换成特定结构的查询语句,在数据库中搜寻,在将查询结果组合成回答呈现给用户。\\
随着深度学习领域的迅速发展,以神经网络为基础的方法已经在文字和图像分类识别领域取得了重大的进展。最近,开始逐渐有人将神经网络应用于比较复杂的任务,例如逻辑推理上。这些通过神经网络产生的隐式的语言表达不需要依赖于特定的语言结构,
也不需要对语言做特定对解析等预处理行为。这些模型在需要很少的预处理过程的前提下,达到或者超过了传统的模型,取得了很好的结果。这些模型依赖于记忆(Memory)模块和注意力(Attention)模块,起到了能够从上下文总结信息、进行推理等作用。\\
Dynamic Memory Network就是这样一个使用了记忆和注意力模块的模型。这个模型在能够取得当前领先的结果。模型由四个模块组成:输入模块、片段记忆模块、问题模块、回答模块。\\
本毕业论文将给予Dynamic Memory Network,首先实现这一模型,然后再对这一模型进行进一步扩展优化,最后将我们优化后的模型应用于技术文档数据集中。\\
\section{目标和内容}
本次毕业论文主要任务是基于Dynamic Memory Network模型的优化及其应用,具体来讲,分为以下几个子目标:
\begin{itemize}
\item 实现基准模型,作为实验对照
\item 实现基础Dynamic Memory Network
\item 对Dynamic Memory Networt进行进一步优化(输入表达可以进一步尝试使用GloVe和基于字的词表达,记忆模块尝试使用GRU、LSTM和bidirectional LSTM优化)
\item 在单词回答的技术上支持多词回答
\item 将优化后的模型应用于技术文档数据集
\end{itemize} 
\section{可行性分析}
从下面几个角度分析,本次毕业论文是可行的:
\begin{itemize}
\item 实验模型:Kumar et al.\cite{DBLP:journals/corr/KumarISBEPOGS15}对这一基础模型的架构已经有了比较清晰的描述。使用主流的机器学习框架Keras和TensorFlow能够相应的减少一些所需要的基础工作量。
\item 数据集:使用Facebook bAbI数据集。数据集是公开可用的,已经被问答系统模型广泛使用和验证,有较为全面的基准可以用于实验的对比。
\item 实验机器:
    \begin{itemize}
    \item Macbook Pro 2016,16GB内存,2.60Hz Intel Core i7:代码编写,初步程序验证。
    \item Precision Tower 7810 Workstation,32GB内存,Intel Xeon i7, Nvidia GeForce GTX 970显卡:程序验证,高阶参数调整。
    \item Google Cloud Platform: 32GB内存,Nvidia Tesla K80显卡:训练数据。 
    \end{itemize}
\end{itemize}
\section{研究方案和关键技术考虑}
\subsection{研究方案}
本次实验采用对照实验,一个基准组和两个对照组。基准组是将词向量直接作为输入进入最简单的神经网络架构中,来实现的。一个实验组是实现Dynmaic Memory Network的基础模型的实验组,
另外一个实验组是对基础Dynamic Memory Network优化后的实验组。
本次实验使用Facebook bABi数据集。bABi数据集分成20个子集,每个子集具有不同的结构,图1.1为数据集中的一些例子\cite{DBLP:journals/corr/WestonBCM15}。每个子集有1000个训练样本和1000个测试样本。我们使用训练样本对我们的三个模型进行训练,
使用测试样本进行测试。一个模型只有能够正确回答95\%的测试集问题,则称之为能够通过这个子任务集。
\begin{figure}[h]
    \centering
    \includegraphics[width=0.6\textwidth]{./images/dataset-example}
      \caption{bABi数据样例}
    \end{figure} 
\subsection{Word2Vec}
Word2Vec(Mikolov et al., 2013\cite{mikolov2013efficient})是一种表达词汇的方法,将一个单词用一个向量来表示,关系比较近的词(例如:king和queen,Shanghai和Beijing)在向量空间相对应的有比较近的距离。如何确定词的关系,一个朴素的想法就是在相似上下文中出现的单词关系比较近。
Word2Vec就是基于这一想法的一种高效的词汇表达方法。特别的,Word2Vec有两种算法,分别是Continuous Bag-of-Words(CBOW)和Skip-Gram。CBOW是根据环境词预测中心词,而Skip-Gram是根据中心词预测环境词。\\
\subsection{GloVe}
尽管Word2Vec已经被实验证明可以发掘出复杂的语义相似关系,但是是根据局部上下文来预测,往往忽略了整体的信息。与之对应的,Pennington et al.\cite{pennington2014glove}提出的Global Vectors for Word Representation (GloVe)使用全局的统计信息预测单词$i$出现在单词$j$的上下文中的概率。
具体来讲,GloVe使用最小二乘作为目标函数训练词词共生矩阵。GloVe在很多词汇相似性任务中已经表现出优越的结果。\\
\subsection{递归神经网络(RNN)}
递归神经网络是一种特别适合对时序信息建模的神经网络架构。在$t$时刻,RNN输入词矩阵$w_t$和上一步的隐式状态向量$h_{t-1}$进行下面的运算后,产生这一步点隐式状态向量:\\
\begin{equation}
h_t = f(Wx_t+Uh{t-1}+b)
\end{equation}
上式中$W$,$U$和$b$是这个RNN的参数。使用RNN对语段建模,能够提取到当前词的前面的所有词的信息考虑在内。
\subsection{GRU}
尽管RNN能够考虑时序关联信息,但是由于使用backpropagation进行优化目标函数时,求导产生的链式相乘,会导致,随着递归向前,产生权重指数级爆炸或消失的问题,难以捕捉长期时间关联。对于权重指数暴炸问题,可以简单的采用超出阈值后截断的方法,就能产生很好的结果。
而Chung et al.\cite{chung2015gated}Gated Recurrent Units(GRU)是一种激活单元,就是为了解决权重指数消失问题,对网络结构进行的优化。下面的一组公式表明了GRU如何根据$h_{t-1}$和$x_t$产生$h_{t}$:\\
\begin{equation}
z_t = \sigma (W^{(z)}x_i + U^{(z)}h_{t-1})
\end{equation}
\begin{equation}
r_t = \sigma (W^{(r)}x_i + U^{(r)}h_{t-1})
\end{equation}
\begin{equation}
\widetilde{h}_t = tanh(r_t \odot Uh_{t-1} + Wx_t)
\end{equation}
\begin{equation}
h_t = (1-z_t) \odot \widetilde{h}_t + z_t \odot h_{t-1}
\end{equation}
式(2)确定了$h_{t-1}$应该多大程度被带入下一阶段。式(3)表明$h_{t-1}$多大程度上影响$\widetilde{h}_t$。式(4)表明新的记忆$\widetilde{h}_t$是由输入$x_t$和$h_{t-1}$确定的。
式(5)表明隐式状态$h_t$最终是由以前的隐式状态$h_{t-1}$和$\widetilde{h}_t$新的记忆组成的。
\subsection{LSTM}
Long Short Term Memories(LSTM)\cite{hochreiter1997long}是另一种与GRU相似的激活单元,与GRU起到的作用相同。下面一组式子表明了LSTM的数学结构:\\
\begin{equation}
i_t = \sigma (W^{(i)}x_t + U^{(i)}h_{t-1})
\end{equation}
\begin{equation}
f_t = \sigma (W^{(f)}x_t + U^{(f)}h_{t-1})
\end{equation}
\begin{equation}
o_t = \sigma (W^{(o)}x_t + U^{(o)}h_{t-1})
\end{equation}
\begin{equation}
\widetilde{c}_t = tanh(W^{(c)}x_t + U^{(c)}h_{t-1})
\end{equation}
\begin{equation}
c_t = f_t \odot c_{t-1} + i_t \odot \widetilde{c}_t
\end{equation}
\begin{equation}
h_t = o_t \odot tanh(c_t)
\end{equation}
式(6)使用输入词和之前的隐式状态来判断输入词是否值得保存,用来产生新记忆。式(7)用来评估过去的记忆是否对于计算现在的记忆有用。
式(8)确定了最终记忆应该多大程度上保存到隐式状态中。式(9)根据输入词$x_t$和$h_{t-1}$产生新的记忆。式(10)根据新记忆和$h_{t-1}$产生最终记忆。

\subsection{Dynamic Memory Network}
Kumar et al.\cite{DBLP:journals/corr/KumarISBEPOGS15} Dynamic Memory Network由四个模块组成,分别是:输入模块、片段记忆模块、问题模块、回答模块。整体的架构如图表所示。\\
\begin{figure}[h]
      \centering
        \includegraphics[width=0.6\textwidth]{./images/dynamic-memory-network-structure}
          \caption{Dynamic Memory Network模型架构}
      \end{figure} 
\subsubsection{输入模块}
输入模块接受一段文字,首先将文字中的每个单词转换为对应的词向量表示,然后将每个单词按顺序依次送入GRU。在每句话的结尾处,输出最终的隐式状态。
片段记忆模块将接收这些隐式状态,并进行总结推理。更加形式化的表示为,对于一个单词序列$T_I$ $w_1,...w_{T_I}$,我们根据下式来更新状态。\\
\begin{equation}
h_t = GRU(L[w_t],h_{t-1})
\end{equation}
然后对于由$T_I$作为子序列组成的序列$s_1,...s_{T_I}$,在每个子序列结束的时候,我们将最终的隐式状态$h_{s_1},...,h_{s_{T_I}}$输出
\subsubsection{问题模块}
问题模块与输入模块相似,接受问题作为输入,将问题序列中的每个单词转换为对应的词向量的表示,然后将每个词向量送入GRU,当所有问题单词都送入后,输出GRU的最终表示。
所以对于问题,形式化的表示为,对于一个包含单词$w_1,...,w_{T_Q}$的问题$T_Q$,我们使用下式更新隐式状态\\
\begin{equation}
h_t = GRU(L[w_t],h_{t-1})
\end{equation}
最后的输出为$h_{T_Q}$
\subsubsection{片段记忆模块}
片段记忆模块对于输入模块在每句话结束时输出的隐式状态和问题模块输出的隐式模块进行总结推理,产生一个最终记忆状态输出给回答模块,用于产生回答。\\
片段记忆模块主要由嵌套的两层GRU组成,内层GRU负责产生片段序列,外层GRU使用问题向量初始化后,根据片段序列产生最终记忆模块作为输入。\\
内层GRU每次通过遍历输入模块的输入序列产生一个片段,在每个片段结束后,内层GRU会把这个片段产生的最终状态送入外层GRU,下面的公式给出了内层GRU更新状态的方法:\cite{xiong2016dynamic}\\
\begin{equation}
z^i_t = [t_t,m,q,c_t \odot q, c_t \odot m,| c_t - q |,|c_t - m|]
\end{equation}
\begin{equation}
Z^i_t = W^{(2)}tanh(W^{(1)}z^i_t+b^{(1)})+b^{(2)}
\end{equation}
\begin{equation}
g^i_t = \frac{exp(Z^i_t)}{\sum_{k=1}^{M_i}exp(Z^i_k)}
\end{equation}
\begin{equation}
h^i_t = g^i_tGRU(c_t,h^i_{t-1} + (1 - g^i_t))h^i_{t-1}
\end{equation}
上面的式子中结合当前状态$c_t$,当前记忆状态$m$和记忆状态$q$来决定当前句子是否值得编码进入回答$g^i_t$中,如果$g^i_t \approx 0$,当前句子将被忽略,现在的状态即为之前的状态。
这个片段的最终状态是当内层GRU遍历晚所有句子之后产生的状态$e^i = h_{T_I}$\\
外层GRU将根据之前的记忆状态和本次片段状态来更新片段状态。\\
\begin{equation}
m^t = GRU(e^t,m^{t-1})
\end{equation}
外层GRU的最终记忆状态将被送入回答模块。
\begin{figure}[h]
      \centering
        \includegraphics[width=0.9\textwidth]{./images/episodic-memory-structure}
          \caption{片段记忆模块架构}
      \end{figure}
\subsubsection{回答模块}
回答模块通过softmax产生对于回答标签的概率分布,然后产生单个单词回答,也可以将输入状态向量RNN来产生多个单词的回答。
\section{预期研究结果}
本次毕设预期结果是实现基础Dynamic Memory Network,对起进行优化,提高回答准确率,在单词回答的技术上支持多词回答,并且最终应用到技术文档中。
\section{进度计划}
\begin{itemize}
\item 3月31日$\sim$4月10日:编写基准模型和基础Dynamic Memory Network模型 
\item 4月11日$\sim$4月17日:训练、验证和测试基准模型和基础Dynamic Memory Network模型
\item 4月18日$\sim$4月25日:编写代码,尝试对Dynamic Memory Network进行优化
\item 4月26日$\sim$4月29日:编写中期报告
\item 4月30日$\sim$5月16日:尝试拓展Dynamic Memory Network,使模型具备多单词回答
\item 5月17日$\sim$5月29日:编写论文                       
\item 5月30日$\sim$答辩前:准备答辩 
\end{itemize}

   %{\sanhao\heiti\filcenter \centerline{本科毕业论文文献综述}}
\section{国内外发展现状}
早期的问答系统例如Baseball\cite{green1961baseball}是针对特定的知识领域所设计的可扩展性比较差,适用范围也比较小。随着互联网的发展,研究人员可用的语料信息的增加。
问答系统的发展得到了极大的提升,先后提出了基于逻辑推理的方法\cite{moldovan2001logic}、基于模版匹配的方法\cite{soubbotin2001patterns}、基于机器学习的方法\cite{yang2002integration}和基于数据冗余的方法\cite{kwok2001scaling}。
这些模型主要利用信息检索或浅层语义理解技术去从大量候选集中寻找答案从而构建智能问答系统,但是检索式问答技术存在一个缺陷,就是答案中一定至少包含一个用户问句中含有的字或者词,但是这在实际情况中往往是不成立的。在此阶段,阻碍问答系统进一步发展的主要困难是高质量的数据集和自然语言处理技术。
随着互联网的进一步发展,高质量的语料得到不断积累。而统计机器学习方法的兴起,自然语言处理技术各个子领域都取得了很大的进步,阻碍问答系统最大的两个问题正在逐步解决。正因如此,很多基于问答系统的产品也逐渐问世,例如:苹果的Siri、谷歌的Google Assistant和微软的Cortana。
\section{研究方向}
实现一个完整的问答系统主要需要三个部分,分别是用户问句的语义理解、信息检索和表示以及对知识进行推理,从而得到最终答案。
\subsection{问句理解}
这个环节实际上要解决的问题是如何将自然语言最准确地转化为计算机可以表示和理解的形式。主要有两种传统的方法来实现,分别是语义解析的方法和基于信息检索的方法。\\
基于语义解析的方法中目前最主流的是组合范畴语法 CCG\cite{kwiatkowski2011lexical}\cite{zettlemoyer2012learning}。CCG首先自然语言问句中的词汇被映射到语义表达式中的词汇,然后按照特定的语法规则将词汇组合起来,进而得到了最终的语义表达式。\\
基于信息检索的方法,首先使用分词、实体识别等NLP技术找到问句中所涉及到的实体和关键词,然后去知识资源库中去进行检索。
\subsection{知识图谱的构建}
知识图谱的构建,主要是将互联网上存在的大量非结构化的文字语料,结构化,并且存储到知识数据库中。在这个过程中,我们需要先对文本进行实体识别,然后对实体进行分类和消歧,接着抽取出关系和事件,这样就能构建一个由实体、关系和事件组成的知识数据库。
目前代表性的工作有:TextRunner\cite{yates2007textrunner}、Wanderlust\cite{akbik2009wanderlust}、NELL\cite{kwiatkowski2011lexical}等。
\subsection{知识推理}
早期的知识推理方法大多对现有知识归纳学习出符号逻辑的推理规则,比如PRA\cite{lao2011random}.这种方法随着推理规则的数量随着关系的数量指数增长,因此很难扩展到大规模知识图谱中去。而随着神经网络等发展,基于Memory和Attention机制等神经网络,在知识归纳推理方面取得了令人瞩目的结果。
这也是我这次毕设的研究基础。\\
\section{关键技术}
\subsection{Word2Vec}
Word2Vec(Mikolov et al., 2013\cite{mikolov2013efficient})是一种表达词汇的方法,将一个单词用一个向量来表示,关系比较近的词(例如:king和queen,Shanghai和Beijing)在向量空间相对应的有比较近的距离。如何确定词的关系,一个朴素的想法就是在相似上下文中出现的单词关系比较近。
Word2Vec就是基于这一想法的一种高效的词汇表达方法。特别的,Word2Vec有两种算法,分别是Continuous Bag-of-Words(CBOW)和Skip-Gram。CBOW是根据环境词预测中心词,而Skip-Gram是根据中心词预测环境词。\\
\subsection{GloVe}
尽管Word2Vec已经被实验证明可以发掘出复杂的语义相似关系,但是是根据局部上下文来预测,往往忽略了整体的信息。与之对应的,Pennington et al.\cite{pennington2014glove}提出的Global Vectors for Word Representation (GloVe)使用全局的统计信息预测单词$i$出现在单词$j$的上下文中的概率。
具体来讲,GloVe使用最小二乘作为目标函数训练词词共生矩阵。GloVe在很多词汇相似性任务中已经表现出优越的结果。\\
\subsection{递归神经网络(RNN)}
递归神经网络是一种特别适合对时序信息建模的神经网络架构。在$t$时刻,RNN输入词矩阵$w_t$和上一步的隐式状态向量$h_{t-1}$进行下面的运算后,产生这一步点隐式状态向量:\\
\begin{equation}
h_t = f(Wx_t+Uh{t-1}+b)
\end{equation}
上式中$W$,$U$和$b$是这个RNN的参数。使用RNN对语段建模,能够提取到当前词的前面的所有词的信息考虑在内。
\subsection{GRU}
尽管RNN能够考虑时序关联信息,但是由于使用backpropagation进行优化目标函数时,求导产生的链式相乘,会导致,随着递归向前,产生权重指数级爆炸或消失的问题,难以捕捉长期时间关联。对于权重指数暴炸问题,可以简单的采用超出阈值后截断的方法,就能产生很好的结果。
而Chung et al.\cite{chung2015gated}Gated Recurrent Units(GRU)是一种激活单元,就是为了解决权重指数消失问题,对网络结构进行的优化。下面的一组公式表明了GRU如何根据$h_{t-1}$和$x_t$产生$h_{t}$:\\
\begin{equation}
z_t = \sigma (W^{(z)}x_i + U^{(z)}h_{t-1})
\end{equation}
\begin{equation}
r_t = \sigma (W^{(r)}x_i + U^{(r)}h_{t-1})
\end{equation}
\begin{equation}
\widetilde{h}_t = tanh(r_t \odot Uh_{t-1} + Wx_t)
\end{equation}
\begin{equation}
h_t = (1-z_t) \odot \widetilde{h}_t + z_t \odot h_{t-1}
\end{equation}
式(2)确定了$h_{t-1}$应该多大程度被带入下一阶段。式(3)表明$h_{t-1}$多大程度上影响$\widetilde{h}_t$。式(4)表明新的记忆$\widetilde{h}_t$是由输入$x_t$和$h_{t-1}$确定的。
式(5)表明隐式状态$h_t$最终是由以前的隐式状态$h_{t-1}$和$\widetilde{h}_t$新的记忆组成的。
\subsection{LSTM}
Long Short Term Memories(LSTM)\cite{hochreiter1997long}是另一种与GRU相似的激活单元,与GRU起到的作用相同。下面一组式子表明了LSTM的数学结构:\\
\begin{equation}
i_t = \sigma (W^{(i)}x_t + U^{(i)}h_{t-1})
\end{equation}
\begin{equation}
f_t = \sigma (W^{(f)}x_t + U^{(f)}h_{t-1})
\end{equation}
\begin{equation}
o_t = \sigma (W^{(o)}x_t + U^{(o)}h_{t-1})
\end{equation}
\begin{equation}
\widetilde{c}_t = tanh(W^{(c)}x_t + U^{(c)}h_{t-1})
\end{equation}
\begin{equation}
c_t = f_t \odot c_{t-1} + i_t \odot \widetilde{c}_t
\end{equation}
\begin{equation}
h_t = o_t \odot tanh(c_t)
\end{equation}
式(6)使用输入词和之前的隐式状态来判断输入词是否值得保存,用来产生新记忆。式(7)用来评估过去的记忆是否对于计算现在的记忆有用。
式(8)确定了最终记忆应该多大程度上保存到隐式状态中。式(9)根据输入词$x_t$和$h_{t-1}$产生新的记忆。式(10)根据新记忆和$h_{t-1}$产生最终记忆。
\subsection{Attention机制}
一句话中的不同的单词的重要性是不同的,例如名词、动词在就会比定冠词介词表达更多的意思。Bahdanau et al\cite{Bahdanauetal.2014}注意到了使用单向量的RNN的最终状态没有办法很好的表征不同输入的部分有不同重要性这一特征。
更近一步的,输出的不同部分可能更看重输入的特定部分。例如,在翻译中,输出的前几个单词通常是根据输入的前几个单词,而输出的后几个单词则更多的依赖于输入的后几个单词。\\
针对这一特点,Attention机制在编码的每一步,首先观察整个输入序列,然后决定在这一个阶段,哪一部分比较重要。\\
Bahdanau et al.提出的模型主要分成两个部分,编码部分和解码部分。编码部分对输入序列编码为向量,解码部分提取编码信息解码成语句,输出语句有一系列单词组成$y_1,...,y_m$。
\subsubsection{编码器}
令$(h_1,...,h_n)$为每个输入句子的隐式向量,这些向量可以使用GRU或者LSTM来产生,包含了词在上下文中的表达信息。
\subsubsection{解码器}
我们可以通过下面的递归式计算隐式状态$s_i$
\begin{equation}
s_i = f(s_{i-1},y_{i_1},c_i)
\end{equation}
在这里$s_{i-1}$是前一个隐式状态,$y_1$是上一个输出单词,上下文向量$c_i$是在解码的第$i$步根据解码状态对原文全文信息等提取表示。\\
对于输入序列中的隐式向量,计算下式得到一个权重值
\begin{equation}
e_{i,j} = a(s_{i-1},h_j)
\end{equation}
上式中$a$可以是任何定义域是$R$的函数,上市会得到一系列值$e_{i,1},...,e_{i,n}$。使用softmax归一化
\begin{equation}
\alpha_{i,j} = \frac{exp(e_{i,j})}{\sum_{k=1}^nexp(e_{i,k})}
\end{equation}
最后对输入隐式序列加权后生成最终的上下文状态向量。
\begin{equation}
c_i = \sum_{j=1}{n}\alpha_{i,j}h_j
\end{equation}
直观的来讲,这个向量能够在第$i$步捕获原文的上下文信息。
\subsection{Dynamic Memory Network}
Kumar et al.\cite{DBLP:journals/corr/KumarISBEPOGS15} Dynamic Memory Network由四个模块组成,分别是:输入模块、片段记忆模块、问题模块、回答模块。整体的架构如图表所示。\\
\begin{figure}[h]
      \centering
        \includegraphics[width=0.6\textwidth]{./images/dynamic-memory-network-structure}
          \caption{Dynamic Memory Network模型架构}
      \end{figure} 
\subsubsection{输入模块}
输入模块接受一段文字,首先将文字中的每个单词转换为对应的词向量表示,然后将每个单词按顺序依次送入GRU。在每句话的结尾处,输出最终的隐式状态。
片段记忆模块将接收这些隐式状态,并进行总结推理。更加形式化的表示为,对于一个单词序列$T_I$ $w_1,...w_{T_I}$,我们根据下式来更新状态。\\
\begin{equation}
h_t = GRU(L[w_t],h_{t-1})
\end{equation}
然后对于由$T_I$作为子序列组成的序列$s_1,...s_{T_I}$,在每个子序列结束的时候,我们将最终的隐式状态$h_{s_1},...,h_{s_{T_I}}$输出
\subsubsection{问题模块}
问题模块与输入模块相似,接受问题作为输入,将问题序列中的每个单词转换为对应的词向量的表示,然后将每个词向量送入GRU,当所有问题单词都送入后,输出GRU的最终表示。
所以对于问题,形式化的表示为,对于一个包含单词$w_1,...,w_{T_Q}$的问题$T_Q$,我们使用下式更新隐式状态\\
\begin{equation}
h_t = GRU(L[w_t],h_{t-1})
\end{equation}
最后的输出为$h_{T_Q}$
\subsubsection{片段记忆模块}
片段记忆模块对于输入模块在每句话结束时输出的隐式状态和问题模块输出的隐式模块进行总结推理,产生一个最终记忆状态输出给回答模块,用于产生回答。\\
片段记忆模块主要由嵌套的两层GRU组成,内层GRU负责产生片段序列,外层GRU使用问题向量初始化后,根据片段序列产生最终记忆模块作为输入。\\
内层GRU每次通过遍历输入模块的输入序列产生一个片段,在每个片段结束后,内层GRU会把这个片段产生的最终状态送入外层GRU,下面的公式给出了内层GRU更新状态的方法:\cite{xiong2016dynamic}\\
\begin{equation}
z^i_t = [t_t,m,q,c_t \odot q, c_t \odot m,| c_t - q |,|c_t - m|]
\end{equation}
\begin{equation}
Z^i_t = W^{(2)}tanh(W^{(1)}z^i_t+b^{(1)})+b^{(2)}
\end{equation}
\begin{equation}
g^i_t = \frac{exp(Z^i_t)}{\sum_{k=1}^{M_i}exp(Z^i_k)}
\end{equation}
\begin{equation}
h^i_t = g^i_tGRU(c_t,h^i_{t-1} + (1 - g^i_t))h^i_{t-1}
\end{equation}
上面的式子中结合当前状态$c_t$,当前记忆状态$m$和记忆状态$q$来决定当前句子是否值得编码进入回答$g^i_t$中,如果$g^i_t \approx 0$,当前句子将被忽略,现在的状态即为之前的状态。
这个片段的最终状态是当内层GRU遍历晚所有句子之后产生的状态$e^i = h_{T_I}$\\
外层GRU将根据之前的记忆状态和本次片段状态来更新片段状态。\\
\begin{equation}
m^t = GRU(e^t,m^{t-1})
\end{equation}
外层GRU的最终记忆状态将被送入回答模块。
\begin{figure}[h]
      \centering
        \includegraphics[width=0.9\textwidth]{./images/episodic-memory-structure}
          \caption{片段记忆模块架构}
      \end{figure}
\subsubsection{回答模块}
回答模块通过softmax产生对于回答标签的概率分布,然后产生单个单词回答,也可以将输入状态向量RNN来产生多个单词的回答。
    \bibliography{data/zjubib}
  %{\sanhao\heiti\filcenter \centerline{本科毕业论文外文翻译}}
文献原文:\\
Kim Y, Jernite Y, Sontag D, et al. Character-aware neural language models[J]. arXiv preprint arXiv:1508.06615, 2015.\\
{\sihao\songti\filcenter \centerline{\textbf{以字为单位的自然语言模型}}}

\section*{摘要}
\addcontentsline{toc}{section}{摘要}
  我们描述了一个只依赖以字为单位输入的自然语言模型。我们的预测依然基于以词为单位。我们的模型对于字使用了
  卷积神经网络(CNN)和高速网络,模型的输出将作为输入,输入到由长短期记忆人工神经网络神经元(LSTM)组成的循环神经网络(RNN)。
  对于Penn Treebank英文数据集,我们的模型比现存的主流模型的参数少60\%,却仍然能取得不相上下多结果。对于内含丰富词法的语言(比如阿拉伯语、捷克语、法语、德语
  西班牙语、俄语),我们的模型的在需要很少的参数的情况下,表现超过LSTM的基准线。我们的结果表明,对于很多语言来说,以字为单位的输入,已经足够对语言建模。
  进一步分析由我们模型的产生的词的表示表明,我们的模型能够基于字编码并且包含语义信息。

\section{介绍}
语言建模是人工智能和自然语言处理最基础的任务。一个语言模型是对于一系列词的概率分布和转换方法。其中,转换方法通常包括通过计数和子序列拟合对马尔科夫链第n次序的假设和估测n-gram的概率。基于计数的模型比较容易训练,但是对于稀有词,可能由于数据稀疏而不能很好的估计。
\\
神经语言模型为了解决n-gram数据稀疏问题通过参数把词向量化(词嵌入),并把向量词作为神经网络的输入。向量化所使用的参数在训练中学习到。
通过神经语言模型获得的词嵌入具有语义上比较接近的词在向量空间比较接近这一性质。\\
虽然神经语言模型的表现优于计数语言模型,但是该模型不能表示语素信息。比如,该模型不能发现相同前缀的词(例如eventfull,eventfully,uneventful,uneventfully)是结构上有关联的,他们应该在向量空间上结构相关。因此,稀有词的嵌入可能被错误的估计。这个问题对于词法丰富的语言来说,问题尤其严重\\
在这篇文章中,我们提出了一个语言模型,这个模型通过以字为单位的卷积神经网络来获取子词信息。模型的输出将作为RNN语言模型的输入。不同于之前使用词素来获取子词信息的模型,我们的模型不需要对词素标记。同样,不同于近期提出的结合词嵌入和字节的模型,我们的模型在输入层不使用词嵌入。因为对于神经语言模型,大多数的参数都是用于生产词嵌入,我们的模型
所需的参数相比于之前的模型显著减少。特别适用于对于模型大小有限制的平台。\\
作为总结,我们的贡献主要有下面几点:\\
\begin{itemize}
\item 对于英语,在Penn Treebank英文数据集上,我们的模型比现存的主流模型的参数少60\%,却仍然能取得不相上下多结果
\item 对于内含丰富词法的语言(比如阿拉伯语、捷克语、法语、德语、西班牙语、俄语),我们的模型的在需要很少的参数的情况下,表现超过LSTM的基准线
\end{itemize}
\section{模型}
Figure 1直观的显示了我们模型的架构。经典的神经语言模型采用词嵌入作为输入,我们的模型采用单层字节max-over-time池化卷积神经网络的输出作为输入\\
作为记号,我们用粗体小写字母标记向量,粗体大写字母标记矩阵,斜体小写字母标记变量,花体大写字母标记集合。为了标记方便,我们假设字和词都已经转换成了索引。\\
\begin{figure}[h]
  \centering
  \includegraphics[width=0.6\textwidth]{./images/figure1}
  \caption{模型架构}
\end{figure}
\subsubsection{递归神经网络}
递归神经网络(RNN)特别适合对于时序模型建模。在每一个时步$t$,根据输入向量$\bf{x}_{i} \in \mathbb{R}^{n}$、隐式状态$\bf{h}_{i}$,应用下面的公式产生下一层隐式状态:\\
\begin{equation}
\bf{h}_{i} = f(\bf(Wx)_{i}+\bf{Uh}_{t-1}+\bf{b})
\end{equation}
在这里$\bf{W} \in \mathbb{R}^{m x n},\bf{U} \in \mathbb{R}^{m x n},\bf{b} \in \mathbb{R}_{m}$都是仿射变换的参数,$f$是非线性对每个元素的函数。理论上,RNN中的隐式状态$\bf{h}_{t}$包含了t个时步的历史信息。然而在实际中,由于权重指数级爆炸或消失的问题,基本的RNN难以捕捉长期时间关联。\\
长短期记忆(LSTM)通过将RNN在每一时步增加一个记忆单元向量$\bf{c}_{i} \in \mathbb{R}_{n}$来解决这一问题。具体来讲,LSTM的每一步利用输入$\bf{x}_{i},\bf{h}_{t-1},\bf{c}_{t-1}$根据下面的公式,计算出$\bf{h}_{t},\bf{c}_{t}$:\\
\begin{equation}
\bf{i}_{t} = \sigma(\bf{W}^{i}\bf{x}_{i}+\bf{U}^{i}\bf{h}_{t-1}+\bf{b}^{i})
\end{equation}
\begin{equation}
\bf{f}_{t} = \sigma(\bf{W}^{f}\bf{x}_{t}+\bf{U}^{f}\bf{h}_{t-1}+\bf{b}^{f})
\end{equation}
\begin{equation}
\bf{o}_{i} = \sigma(\bf{W}^{o}\bf{x}_{t}+\bf{U}^{o}\bf{h}_{t-1}+\bf{b}_{o})
\end{equation}
\begin{equation}
\bf{g}_{t} = tanh(\bf{W}^{g}\bf{x}_{t}+\bf{U}^{g}\bf{h}_{t-1}+\bf{b}_{g})
\end{equation}
\begin{equation}
\bf{c}_{t} = \bf{f}_{t}\odot\bf{c}_{t-1}+\bf{i}_{t}\odot\bf{g}_{t}
\end{equation}
\begin{equation}
\bf{h}_t = \bf{o}_{t}\odot tanh(\bf{c}_{t})
\end{equation}
在这里$\sigma(\cdot)$和$tanh(\cdot)$是对每个元素的sigmoid函数和双曲正切函数,$\odot$是元素间相乘运算符,$\bf{i}_{t},\bf{f}_{t},\bf{o}_{t}$分别是输入门、遗忘门、输出门。当$t=1$时,$\bf{h}_{0}$和$\bf{c}_{0}$被初始化为零向量。LSTM需要的参数是$\bf{W}_{j},\bf{U}_{j},\bf{b}_{j},j\in\{i,f,o,g\}$\\
LSTM中的记忆单元是随时间变化的,从而缓解了权重指数消失问题。虽然权重指数爆炸仍然是一个问题,但是在实际训练过程中,一些简单的优化策略(例如变化率截断)效果很好。LSTM在包括语言建模在内的很多任务中显示出比基本RNN更好的结果。通过使$\bf{h}_{t}$作为$t$时的输入,可以很容易的将RNN/LSTM扩展到多层。实际上,在很多任务中,使用多层网络是获得有竞争力结果的关键选择\\
\subsection{Recurrent Neural Network Language Model}
记$\mathscr{V}$为固定的词汇集大小。给定历史序列$w_{1:t}=[w_{1},...,w_{t}]$,一个语言模型确定了$w_{t+1}$的概率分布。一个循环神经网络语言模型(RNN-LM)通过对隐式层应用仿射变换然后再应用softmax函数来实现这一点:\\
\begin{equation}
Pr(w_{t+1}=j\mid w_{1:t}) = \frac{exp(\bf{h}_{t}\cdot \bf{p}^{j}+q^{j})}{\sum_{j^{\prime} \in \mathscr{V}} exp(\bf{h}_{t}\cdot\bf{p}^{j^{\prime}}+q^{j^{\prime}})}
\end{equation}
在这里$\bf{p}^{j}$是$\bf{P} \in {\mathbb{R}}^{m \times |  \mathscr{V} |}$的第$j$列,$q^j$是偏差项。同样的,传统的RNN语言模型把词作为输入,如果$w_{t} = k$,传统模型在时步$t$的输入是$\bf{X} \in \mathbb{R}^{n \times | \mathscr{V} |}$的第$k$列$x^{k}$。
我们的模型把词嵌入矩阵$\bf{X}$替换成了下面描述的字为单位的卷积神经网络的输出。\\
我们记$w_{1:T}=[w_{1},...,w_{T}]$为训练词库的词序列,训练包括最小化该序列的负log概率(NLL),这一过程通常有啊截断反向传播来实现
\begin{equation}
NLL = - \sum_{t=1}^{T}logPr(w_t \mid w_{1:t-1})
\end{equation}
\subsubsection{字单位的卷积神经网络}
在我们的模型中,$t$时刻的输入是字单位的卷积神经网络的输出(CharCNN),我们将在这里小节里面描述CharCNN。CNN在计算机视觉领域的应用已经取得了突出的效果,并且对于多种自然语言处理(NLP)任务也是十分有效的。
针对NLP应用CNN有不同的架构,因为NLP需要时序卷积而不是空间卷积。\\
令$\mathscr{C}$为字的集合,$d$为字嵌入的维度,$\bf{Q} \in \mathbb{R}^{d \times | \mathscr{C} |}$为字嵌入矩阵。假设词$k \in \mathscr{V}$是由字序列$[c_1,...,c_l]$组成,$l$是词$k$的长度。
$k$的字表示由矩阵$\bf{C}^k \in \mathbb{R}^{d \times l}$给出,该矩阵的第$j$列是字$c_j$的字嵌入\\
我们对$\bf{C}^k$和宽度为$w$的过滤器$\bf{H} \in \mathbb{R}^{d \times w}$应用窄卷积。然后,我们加上偏差,再应用一个非线形函数来获得一个特征图谱$\bf{f}^k \in \mathbb{R}^{l - w + 1}$。$\bf{f}^k$的第$i$项由下式得出:\\
\begin{equation}
\bf{f}^k[i] = tanh(\langle \bf{C}^k[*,i:i+w-1],\bf{H}\rangle + b)
\end{equation}
在这里$bf{C}^k[*,i:i+w-1]$是$\bf{C}^k$的$i$到$i+w-1$列,$\langle \bf{A},\bf{B} = Tr(\bf{AB}^T)\rangle$是Frobenius内积。最后我们应用max-over-time作为过滤器$\bf{H}$对应的特征\\
\begin{equation}
y^k = max_i\bf{f}^k[i]
\end{equation}
这样做的目的是对于一个给定的过滤器,获取最重要的特征,也就是具有最高值的特征。一个过滤器本质上是选出一个字$n$元语法,$n$的大小由过滤器宽度决定。\\
我们通过一个特征由一个过滤矩阵得出描述了这个过程。我们的CharCNN使用多个不同宽度的过滤器获得$k$的特征向量。所以如果我们有$h$个过滤器 $\bf{H}_1,...\bf{H}_h$,就有$k$的输入表示$\bf{y}^k = [y^k_1,...,y^k_h]$,
对于很多NLP应用$h$通常在$[100,1000]$这个范围内选择。\\
\subsection{高速网络}
我们只需要在RNN-LM中在t时刻,用$\bf{y}^k$替换$\bf{x}^k$,正如我们稍后将要证明的,这样就已经能有很高效的表示了。我们也可以对$\bf{y}^k$使用一个多层感受器(MLP),来建模过滤器选择的n元语法的交集,但是我们发现这样的表现不是很好。\\
我们通过将$\bf{y}^k$运行在高速网络上进一步提高的我们的表现。高速网络是一层的MLP应用了仿射变换之后再施加一个非线形函数,从而得到一组新的特征。\\
\begin{equation}
\bf{z} = g(\bf{Wy} + \bf{b})
\end{equation}
一层高速网络实现了做了下面的过程:\\
\begin{equation}
\bf{z} = \bf{t} \odot g(\bf{W}_H\bf{y}+\bf{b}_H) + (\bf{1} - \bf{t})\odot \bf{y}
\end{equation}
在这里$g$是一个非线性函数,$\bf{t} = \sigma(\bf{W}_T\bf{y}+\bf{b}_T)$被称作转换门,$ (\bf{1} - \bf{t})$被称作进位门。
与LSTM网络中的记忆单元相似,高速网络会使一些维度的输入直接作为输出。在构建的时候,$\bf{y}$和$\bf{z}$的维度必须相同,$\bf{W}_T$和$\bf{W}_T$都是方阵。

\section{实验设置}
根据语言建模的标准,我们使用复杂度(PPL)来衡量我们模型的表现。对于一个序列$[w_1,...,w_T]$,一个模型的复杂度由下式给出:\\
\begin{equation}
  PPL = exp(\frac{NLL}{T})
\end{equation}
在这里NLL是对测试集来计算的。我们针对不同的语言和不同大小的词汇集进行了测试。\\
针对Penn Treebank(PTB)数据集,我们进行了高阶参数查询、模型检查和分离学习。使用0-20进行标准学习、21-22进行验证、23-24进行测试。PTB有一百万个标签,$| \mathscr{V} | = 10k$,已经被广泛的用于语言建模。\\
在针对PTB数据集优化高阶参数之后,我们将模型应用于很多词法丰富的语言:捷克语、德语、法语、西班牙语、俄语、阿拉伯语。尽管未经处理的数据是公开可用的,我们从作者那里获取到了预处理过的数据,他们的NLM充当了我们模型的基准。我们在每个语言有一百万个标签的小数据集(DATA-S)和$| \mathscr{V} | = 10k$远大于PTB的大数据集(DATA-L)上都进行了训练。\\
在这些数据集中,只有只出现一次的词汇会被<unk>替换,所以我们有效的使用了数据集上包含的词汇集。\\
\begin{figure}[h]
  \centering
  \includegraphics[width=0.6\textwidth]{./images/table1}
  \caption{词汇集统计数据}
\end{figure}


\subsection{优化}
我们使用截断反向传播来训练我们的模型。我们使用stochastic gradient descent反向传播35个时步。学习率初始设为1.0,当PPL没有在每个epoch在验证集下降超过1.0时,学习率变为原来的一半。\\
在DATA-S上,我们令batch大小为20,在DATA-L中我们使用batch大小为100。变化率是每个batch的平均。我们对于非阿拉伯语训练25个epoch,对于阿拉伯语训练30个epoch,选出在验证集上表现最好的模型。模型随机初始化为服从[-0.05,0.05]上的均匀分布。\\
我们在LSTM从输入到隐式层和softmax到输出层使用概率为0.5的丢弃法进行正规化学习。我们进一步限制变化率的范数小于5。如果$L_2$规范化的变化率超过5,在更新之前,我们设置他为5。变化率范数限制对于训练模型是很重要的。我们的选择很大程度上是受Zaremba et al.做的工作的启发。\\
最后,为了加速DATA-L上的学习,我们使用了hierarchical softmax。Hierarchical softmax是一个训练$| \mathscr{V} |$大小不同的语言模型的常用策略。我们选择$c = \lceil \sqrt{| \mathscr{V} |}  \rceil$个簇,随机的把$\mathscr{V}$分成大小相同的子集$\mathscr{V}_1,...,\mathscr{V}_c$,这样就有:\\
\begin{equation}
Pr(w_{t+1} = j \mid w_{1:t}) = \frac{exp(\bf{h}_t \cdot \bf{s}^r + t^r)}{\sum_{r^\prime = 1}{c} exp(\bf{h}_t \cdot \bf{s}^{r^\prime} + t^{r^\prime})} \times \frac{exp(\bf{h}_t \cdot \bf{p}^j_r + q^j_r)}{\sum_{j^\prime \in \mathscr{V}_r}exp(\bf{h}_t \cdot \bf{p}_r^{j^\prime} + q_r^{j^\prime})}
\end{equation}

在这里$r$是簇的索引,有$j \in \mathscr{V}_r$。上式的第一项是选择第$r$簇的概率,第二项是已知第$r$簇被选的情况下选中词$j$的概率。

\section{实验结果}
\subsection{English Penn Treebank}
\begin{figure}[h]
  \centering
  \includegraphics[width=0.6\textwidth]{./images/table2}
  \caption{小模型和大模型的架构}
\end{figure}
我们为了评估性能和大小的取舍,对我们的模型训练了两个版本。小模型(LSTM-Char-Small)和大模型(LSTM-Char-Large)的架构在表2中进行了总结。\\
和其他基准一样,我们也用词嵌入训练了两个比较模型(LSTM-Word-Small,LSTM-Word-Large)。LSTM-Word-Small使用了200个隐式单元。LSTM-Word-Large使用了650个隐式单元。词嵌入的大小分别是200和650.这些大小的选择是为了保持和字单位模型相似的参数数目。\\
正如从表3中看到的,我们的模型在参数少60\%的情况下,与现在领先的模型取得差不多的表现。我们的小模型明显优于其他大小相似的NLM。\\
\begin{figure}[h]
  \centering
  \includegraphics[width=0.6\textwidth]{./images/table3}
  \caption{小模型和大模型的架构}
\end{figure}

\subsection{其他语言}
我们模型在English PTB上的优异表现表明可以使用它来处理更大规模大数据。然而,英语从词法角度讲是相对简单的,所以我们下一部分的结果主要集中在有更丰富词法的语言上。\\
我们将我们的结果同morphological logbilinear (MLBL) 模型相比较。MLBL通过在输入和输出阶段的语素嵌入也考虑了子词信息。因为在同MLBL比较时,我们使用的LSTM比MLBL使用的feed-forward有众所周知的更优异的表现,我们又训练了一个LSTM版本的语素NLM。这个版本中,传入LSTM的词表示是词的语素嵌套的和。具体来讲,假设$\mathscr{M}$是一个语言的语素集,
$\bf{M} \in \mathbb{R}^{n \times | \mathscr{M} |}$是语素嵌入矩阵,$\bf{m}^j$是$\bf{M}$的第$j$列,给定词$k$,下式为输入LSTM的词的表示:\\
\begin{equation}
\bf{x}^k + \sum_{j \in \mathscr{M}^k}\bf{m}^j
\end{equation}
在这里$\bf{x}^k$是词嵌入,$\mathscr{M}_k \subset \mathscr{M}$是词$k$的词素集。词素通过在预处理阶段运行一个词素标记非监督算法来获得。词嵌入本身是在词素嵌入的基础上的。词素嵌入对于小模型和大模型的大小分别是200/650。我们进一步训练词单位的LSTM作为另一个基准.\\
从表4中可以看出,我们的字单位的模型表现明显优于词单位的其他模型,尽管我们的模型更小。字模型同样优于词素模型(MLBL和LSTM架构),注意到词素模型需要更多的参数,因为词嵌入作为输入的一部分。\\
由于内存限制,我们在DATA-L上只训练了小模型。有趣的是,在西班牙语、法语和英语中,我们没有观察到词模型和语素模型显著的区别。字模型同样比词和语素取得更好多结果。在英语训练中,对于$\mathscr{V}$很大大时候,我们同样也观察到复杂度显著的减少。作为这一小节的总结,我们要指出,对于不同语言,我们使用同样的架构,并且对于高阶参数,我们并没有针对特定的语言有特定的参数调整\\

\begin{figure}[h]
  \centering
  \includegraphics[width=0.6\textwidth]{./images/table4}
  \caption{对于DATA-S在测试集上的复杂度}
\end{figure}
\begin{figure}[h]
  \centering
  \includegraphics[width=0.6\textwidth]{./images/table5}
  \caption{对于DATA-L在测试集上的复杂度}
\end{figure}

\section{讨论}
\subsection{学习到的词表示}
\begin{figure}[h]
  \centering
  \includegraphics[width=0.95\textwidth]{./images/table6}
  \caption{最近邻词在字单位和词单位经过高速网络后的表达}
\end{figure}
我们研究了模型在PTB数据集上学习到的词的表示。表6显示了词单位和字单位模型学习到的近邻词表示。对于字单位模型,我们比较了使用高速网络前后的词表示。\\
在进入高速网络层之前,词表示看起来只依赖于表面形式,例如单词you的最近邻是your,young,four,youth。高速网络似乎能够编码和拼写无关的语义信息。
在经过高速网络之后,you的最近邻是we,we的拼写和you差很多。另一个例子是,while和though,他们看起来拼写很不一样,但是我们的模型能够把他们放置到临近的地方。
我们的模型也犯了很多明显的错误(比如his和hhs),尽管可能是由于数据集太小造成的,还是揭示了我们模型的一些局限。\\
\begin{figure}[h]
  \centering
  \includegraphics[width=0.6\textwidth]{./images/figure2}
  \caption{字的n元语法的点图}
\end{figure}
\subsection{学习到的字n元语法表示}
正如前面讨论到的,CharrCNN的每个筛选器对于识别特定字的n元语法是必要的。我们最初的期待是每个筛选器能够学习在不同的语素上激活,然后根据识别出的语素实现语义表达。
然而,通过研究筛选器选出的n元语法,我们发现它们并没有和合法的语素建立起联系。\\
为了能够建立起更好的关于字模型学了的什么的直觉,通过主成分分析,我们绘出了学习到的n元语法的表示。我们把每个字n元语法送入CharCNN,然后利用CharCNN的输出作为对应的字的固定维度的n元语法表示。
正如图表2显示的,我们的模型学习到了如何区分前缀(红)、后缀(蓝)和其他(灰)。我们也发现这个表达对于包含连字符的字n元语法表示特别敏感,我们的推测是因为连字符通常预示着这个词是演讲的一部分。\\

\subsection{高速网络层}
我们通过隔离分析定量的研究高速网络的作用。我们移除模型的一层高速网络层,发现模型的性能严重下降。因为性能的不同可能是由于模型大小减少所造成的,
我们也训练了一个将$\bf{y}^k$送入一个一层MLP,然后将它作为LSTM的输入。尽管可能是因为优化的问题,但是我们发现MLP效果也很差。\\
我们推测高速网络对于CNN来说是必要的。CNN已经在很多NLP任务中被证明是有效的方法。我们推测将高速网络结合进CNN中将会取得更好的结果。\\
我们也发现:(1)有一到两层高速网络是很重要的,但是进一步增加高速网络层数,并没有取得更优的结果。(2)在最大池化前设置更多的卷积网络并没有帮助。(3)高速网络对于使用词嵌入作为输入的模型并没有帮助。\\
\begin{figure}[h]
  \centering
  \includegraphics[width=0.6\textwidth]{./images/table7}
  \caption{小模型和大模型在使用/不使用高速网络的复杂度对比}
\end{figure}
\subsection{词汇量大小的影响}
我们接着学习了词汇集的大小对于不同模型结果的影响。我们使用DATA-L中的德语(DE)数据集,实用不同的词汇集大小,计算从词单位和字单位复杂率减少的大小。为了控制词汇量的变化,我们选出最常用的$k$个词,把他们替换为<unk>。
因为在前面的实验中,字模型没有使用<unk>的表面形式,只是把它当成另外一个标签。尽管表8表明随着词汇量的增加,复杂度减少的没有那么明显,我们还是发现字模型在所有的场景下都还是优于词模型。\\
\begin{figure}[h]
  \centering
  \includegraphics[width=0.6\textwidth]{./images/table8}
  \caption{不同的词汇量下,字模型比词模型在复杂度减少量}
\end{figure}
\subsection{进一步观察}
我们在进一步的的实验和观察中发现了下面的情况:
\begin{itemize}
\item 结合词嵌入和CharCNN的输出的词表示结果稍微糟糕一些。这个结果很让我们意外,因为有相关报道表明这样的结合在演讲标记和实体识别中有明显的改善。尽管我们的实验结果可能是 由于我们的实验设置的不够合理,
但是还是表明,对于一些任务,词嵌入式无用的,字输入就已经足够了
\item 尽管我们的模型需要卷积操作,所以相对于只是进行简单词查找的词模型慢一些,但是我们还是发现这种速度的区别可以通过GPU实现的优化来控制的。比如,在PTB数据集上,大的字单位模型训练速度为1500标签每秒,相对应的,
词模型训练速度为3000标签每秒。为了评分,因为CharCNN的输出可以被预先训练出来,所以我们的模型可以和词模型有相同的运行时间。然而,这样会增加模型的大小,所以这是一个在运行时速度和内存之间的取舍问题。
\end{itemize}
\section{总结}
我们引入了只使用字单位输入的神经语言模型。预测还是在词单位进行的,尽管模型有更少的参数,我们的模型表现超出使用词单位和使用语速单位的模型的基准线。
我们的模型对于词嵌入作为输入在神经语言建模中的必要性。\\
对于从字模型中获取到的词的表示进行分析之后揭示了这个模型能够只从字编码丰富的语义信息和拼写信息。使用CharCNN和高速网络层对于表达学习还是有很大帮助的。\\
到目前为止,由于序列化处理词作为自然语言处理比较普世的方法,如果这篇论文的方法能够对其他任务有所帮助是很好的。\\
\section{致谢}
我们非常感谢Jan Botha,他为我们提供了预处理好的数据和模型的结果。

\includepdf[pages=-]{./data/translate-paper-english.pdf}

%==============================================================
%这也是个不需要自己修改的部分。

  %\backmatter %结束章节自动编号

%==============================================================
\end{document}
